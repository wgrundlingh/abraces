% $Id: abraces-doc.tex,v 2.0 2021/03/31 00:00:00 wgrundlingh stable $
\RequirePackage{xcolor}% http://ctan.org/pkg/xcolor (required to "overload" color package)
\documentclass[10pt]{ltxdockit}[2011/03/25]

%\usepackage{xspace}% Already loaded by ltxdockit
\usepackage[overload]{abraces}
\usepackage{booktabs,shortvrb,amsmath,xfp}
\MakeShortVerb{\|}

\rcsid{$Id: abraces.tex,v 2.0 2021/03/31 00:00:00 wgrundlingh stable $}
%\setcounter{secnumdepth}{2}
%\setcounter{tocdepth}{4}

% https://tex.stackexchange.com/a/472150/5764
% Use cases for breaking a paragraph that is not visible to the end user, forcing the paragraph to be set with a full-width/-length
% last line. Then one could insert a \pagebreak if needed, depending on the layout requirements
\newcommand{\parnopar}[1][]{\parfillskip=0pt\par%
  #1%
  \parskip=0pt\noindent\parfillskip=0pt plus1fil}

\newsavebox{\codebox}% For storing code segments in footer.
\newcommand*{\abracesctan}{http://ctan.org/pkg/abraces}

\lstset{
  language={[LaTeX]TeX},
  morekeywords={%
    aoverbrace,
    aunderbrace,
    color,
    bracetext,
    bracescript,
    FnD,
    newcolumntype,
    newbracespec,
    bracecolor,
    color,
    textcolor,
    aftergroup,
    text}
}

\titlepage{%
  title={The \sty{abraces} package},
  subtitle={Asymmetric or arbitrary braces},
  url={\abracesctan},
  author={Werner Grundlingh},
  email={latex.abraces@gmail.com},
  revision={\rcsrevision},
  date={\rcstoday}}

\hypersetup{%
  pdftitle={The abraces package},
  pdfsubject={Asymmetric or arbitrary braces},
  pdfauthor={Werner Grundlingh (latex.abraces@gmail.com)},
  pdfkeywords={tex, latex, braces, overbrace, underbrace, math-mode}}

\usepackage{parskip}

\newlength{\bracecusplen}
\settowidth{\bracecusplen}{$\braceru\bracelu$}

\begin{document}

\printtitlepage
%\tableofcontents

\section{Introduction} \label{sec:introduction}

The \href{http://ctan.org/pkg/abraces}{\sty{abraces} package} provides a character key-driven interface to supplement new constructions of the traditional \lstinline!\overbrace! and \lstinline!\underbrace! pairs in an asymmetric or arbitrary way.

\section{Basic user interface} \label{sec:user-interface}

This package defines two counterparts to the existing braces:

\begin{ltxsyntax}

\cmditem{aoverbrace}[brace spec]{stuff}[script spec]\lstinline!^{!\prm{upper script}\lstinline!}_{!\prm{lower script}\lstinline!}!

\cmditem{aunderbrace}[brace spec]{stuff}[script spec]\lstinline!^{!\prm{upper script}\lstinline!}_{!\prm{lower script}\lstinline!}!

\end{ltxsyntax}

Note that both the \prm{brace spec} and \prm{script spec} arguments are optional, as well as the use of an \prm{upper script} and \prm{lower script}. As such, in its most basic form, \lstinline!\aoverbrace{!\prm{stuff}\lstinline!}! (and \lstinline!\aunderbrace{!\prm{stuff}\lstinline!}!) would be similar to the traditional \lstinline!\overbrace{!\prm{stuff}\lstinline!}! (and \lstinline!\underbrace{!\prm{stuff}\lstinline!}!). However, if you specify a \prm{brace spec} -- a construction pattern based on the elements in \hyperref[tab:abrace-spec]{Table~\ref*{tab:abrace-spec}} -- you could adjust the shape of the brace in an arbitrary way. These definitions are robust.

The \prm{brace spec} interface is based on a ratio-principle, allowing one to put a larger share of ``filler'' (the horizontal rule) at any location within the brace construction. The traditional \lstinline!\overbrace! and \lstinline!\underbrace! pairs have a \mbox{1:1} share between the left and right side (either side of the tip/cusp of the brace), thereby forcing the tip/cust to be placed directly in the center horizontally. With \lstinline!abraces!, using a \mbox{1:2} ratio would place the brace cusp one third (from the left) into the brace. Similary a \mbox{3:2} ratio would place the cusp 40\% (or two fifths) from the right edge of the brace. The same holds for elements specified within \prm{script spec}, except these are used to alter the location of the scrips. For more detail, see section \hyperref[sec:advanced-interface]{\ref*{sec:advanced-interface}~\nameref*{sec:advanced-interface}}.

Other, more complex constructions are possible by mixing the elements presented in \hyperref[tab:abrace-spec]{Table~\ref*{tab:abrace-spec}}. See section \hyperref[sec:examples]{\ref*{sec:examples} \nameref*{sec:examples}} for a showcase of uses.

\begin{ltxsyntax}

\cmditem{newbracespec}{char}{brace spec}

\end{ltxsyntax}

This allows the user to define a new brace specification \prm{char} that results in the (possibly complex) construction \prm{brace spec}. Note that \prm{char} should be different from any already used (see \hyperref[tab:abrace-spec]{Table~\ref*{tab:abrace-spec}}). The usage is similar to that of a \lstinline!\newcolumntype! construction provided by the \href{http://ctan.org/pkg/array}{\sty{array} package}.

\begin{table}[t]
  \centering
  \makeatletter
  \begin{tabular}{cl}
    \toprule
          \prm{spec} character      & Output \\
    \midrule
                  |l|               & $\bracelu$ (left upward brace) \\
                  |L|               & $\braceld$ (left downward brace) \\
                  |r|               & $\braceru$ (right upward brace) \\
                  |R|               & $\bracerd$ (right downward brace)\\
                  |U|               & $\braceru\bracelu$ (upward cusp) \\
                  |D|               & $\bracerd\braceld$ (downward cusp) \\
           |,[|\prm{len}|]|         & % Similar to \downbracketend in abraces.sty
                                      \sbox\z@{$\braceld$}\dimen@\ht\z@ \advance\dimen@.7\fontdimen5\textfont2\relax
                                      \vrule \@width\ht\z@ \@height\ht\z@ \@depth\dimexpr\dimen@-\ht\z@\ (downward end with optional \prm{len}gth control) \\
           |'[|\prm{len}|]|         & % Similar to \upbracketend in abraces.sty
                                      \sbox\z@{$\bracelu$}\dimen@\ht\z@ \advance\dimen@.7\fontdimen5\textfont2\relax
                                      \vrule \@width\ht\z@ \@height\dimen@ \@depth\z@\ (upward end with optional \prm{len}gth control) \\
                  |0|               & (single) Empty fill \\
           |1|, \ldots, |9|         & Copies of regular fill \rule[.4ex]{2em}{1.5pt} \\
          |@{|\prm{stuff}|}|        & Places \prm{stuff} into brace \\
           |!{|\prm{len}|}|         & Regular fill of length \prm{len} \\
    |*{|\prm{num}|}{|\prm{stuff}|}| & Repeat \prm{stuff} a total of \prm{num} times \\
    \bottomrule
  \end{tabular}
  \makeatother
  \caption{Character specifications within \prm{brace spec} used to construct braces.}
  \label{tab:abrace-spec}
\end{table}

By default, any scripts will be placed at the cusp (|D| when providing a \prm{lower script} via \lstinline|_| or |U| when providing an \prm{upper script} via \lstinline|^|) of the brace. If you use more than one cusp within your \prm{brace spec}, you can separate scripts using \lstinline!&! which will place them above/below subsequent brace cusps, similar to separating columns within a \lstinline!tabular!.

\begin{ltxsyntax}

\cmditem{bracecolor}{spec}

\end{ltxsyntax}

If you're interested in using any form of colour, \lstinline|\bracecolor| will allow you to change the brace colour via an \lstinline!@!-insertion (for example, \lstinline!@{\bracecolor{red}}! would yield a \textcolor{red}{red} brace from that point onward). Regular script colouring can still be achieved using \lstinline!\color! or \lstinline!\textcolor!. The motivation here is that elements within the \lstinline!@!-insertions are grouped; \lstinline!\bracecolor! uses \lstinline!\aftergroup! to re-insert the use of \lstinline!\color!.

If the package is loaded with the \lstinline!overload! option
\begin{lstlisting}
\usepackage[overload]{abraces}
\end{lstlisting}
the traditional \lstinline!\overbrace!/\lstinline!\underbrace! pairs are redefined to be equivalent to \lstinline!\aoverbrace! and \lstinline!\aunderbrace!, respectively, via a straight-forward \lstinline!\let!:
\begin{lstlisting}
\let\overbrace\aoverbrace
\let\underbrace\aunderbrace
\end{lstlisting}

\section{Advanced uses} \label{sec:advanced-interface}

There are some cases where you don't want to place elements exactly at the tip(s) of the brace(s). Maybe it's because you don't have any tips on your braces, or need to write other information around the tips. To that end, the second optional \prm{script spec} argument is provided where you can specify a new sequence that may differ from the original brace specification. Without any tips, for example, one could write \lstinline!$\aoverbrace[L1R]{a + b + \cdots +!\parnopar\lstinline!z}[U]^{\text{26 terms}}$! to denote a grouping of elements:
\[
  \aoverbrace[L1R]{a + b + \cdots + z}[U]^{\text{26 terms}}
\]
More detailed examples are covered in section \hyperref[sec:examples]{\ref*{sec:examples}~\nameref*{sec:examples}}.

\section{Examples} \label{sec:examples}

Some basic examples of the types of braces that can be constructed using \sty{abraces}:
\begin{lstlisting}
\newcommand{\FnD}{%
  \textrm{The quick brown fox jumped over the lazy dog}}
\end{lstlisting}

\newcommand{\FnD}{%
  \textrm{The quick brown fox jumped over the lazy dog}}
\begin{itemize}
  \item \lstinline!\aoverbrace{\FnD}! (traditional \lstinline!\overbrace!): \\
    $\aoverbrace{\FnD}$
    
  \item \lstinline!\aunderbrace{\FnD}! (traditional \lstinline!\underbrace!): \\
    $\aunderbrace{\FnD}$
    
  \item \lstinline!\aoverbrace[L3U1R]{\FnD}!: \\
    $\aoverbrace[L3U1R]{\FnD}$
    
  \item \lstinline!\aoverbrace[*{6}{0}l1D1r*{5}{0}]{\FnD}!: \\
    $\aoverbrace[*{6}{0}l1D1r*{5}{0}]{\FnD}$
    
  \item \lstinline~\aunderbrace[l2D1r000@{\bracecolor{blue!70!black}}l1D2r]{\FnD}~: \\
    $\aunderbrace[l2D1r000@{\bracecolor{blue!70!black}}l1D2r]{\FnD}$
    
  \item \lstinline!\aunderbrace[l1D2U2D1r]{\FnD}!: \\
    $\aunderbrace[l1D2U2D1r]{\FnD}$
    
  \item \lstinline!\aoverbrace[L1R]{\FnD}!: \\
    $\aoverbrace[L1R]{\FnD}$
    
  \item \lstinline!\aunderbrace[L1U3R]{\FnD}!: \\
    $\aunderbrace[L1U3R]{\FnD}$
    
  \item \lstinline!\aunderbrace['6,0l3D3r0,6']{\FnD}!: \\
    $\aunderbrace['6,0l3D3r0,6']{\FnD}$
    
  \item \lstinline!\aoverbrace[L5*{3}{01}05U50*{3}{10}5R]{\FnD}!: \\
    $\aoverbrace[L5*{3}{01}05U50*{3}{10}5R]{\FnD}$
    
  \item \lstinline!\aunderbrace[l1@{\hspace{5em}}2D2@{~\ldots~}1r]{\FnD}!: \\
    $\aunderbrace[l1@{\hspace{5em}}2D2@{~\ldots~}1r]{\FnD}$
    
  \item \lstinline~\aunderbrace[l1R@{\bracecolor{red!80!white}}L1r]{\FnD}~: \\
    $\aunderbrace[l1R@{\bracecolor{red!80!white}}L1r]{\FnD}$
    
  \item \lstinline~\aoverbrace[,1D!{5em},]{\FnD}~: \\
    $\aoverbrace[,1D!{5em},]{\FnD}$
\end{itemize}

Some more advanced techniques of adding content to the brace cusps:
\begin{itemize}
  \item \lstinline!\aoverbrace[L1U2R]{\FnD}^{\text{one-third of the way}}!: \\
    $\aoverbrace[L1U2R]{\FnD}^{\text{one-third of the way}}$
  
  \item \lstinline!\aoverbrace[L1U1D1U1R]{\FnD}^{\text{left} & \text{right}}!: \\
    $\aoverbrace[L1U1D1U1R]{\FnD}^{\text{left} & \text{right}}$

  \item \lstinline!\aoverbrace[L1U1D1U1R]{\FnD}! \\
  \quad\lstinline![L1U1U1U1R]^{\text{left} & \text{middle} & \text{right}}!: \\
    $\aoverbrace[L1U1D1U1R]{\FnD}[L*{3}{1U}1R]^{\text{left} & \text{middle} & \text{right}}$

  \item
    \newcommand{\ind}[1][1]{\hspace*{\fpeval{#1*1.5}em}}%
    \lstinline!\newbracespec{u}! \\
    \ind\lstinline!{@{\hspace{-.5\bracecusplen}}U@{\hspace{-.5\bracecusplen}}}! \\
    \lstinline!\newbracespec{d}! \\
    \ind\lstinline!{@{\hspace{-.5\bracecusplen}}D@{\hspace{-.5\bracecusplen}}}! \\
    \lstinline!$\aunderbrace! \\
    \ind[1]\lstinline![00l2@{\hspace{-\bracecusplen}}1r]{% \aunderbrace brace script! \\
    \ind[2]\lstinline!\aoverbrace! \\
    \ind[3]\lstinline![L1@{\hspace{-\bracecusplen}}1R000]% \aoverbrace brace script! \\
    \ind[3]\lstinline!{\FnD}% stuff! \\
    \ind[3]\lstinline![1u13]% \aoverbrace script spec! \\
    \ind[3]\lstinline!^{2/5}% \aoverbrace upper script! \\
    \ind\lstinline!}! \\
    \lstinline![43d3]% \aunderbrace script spec! \\
    \lstinline!_{3/5}$% \aunderbrace lower script!: \\
    \newbracespec{u}{@{\hspace{-.5\bracecusplen}}U@{\hspace{-.5\bracecusplen}}}%
    \newbracespec{d}{@{\hspace{-.5\bracecusplen}}D@{\hspace{-.5\bracecusplen}}}%
    $\aunderbrace
      [00l2@{\hspace{-\bracecusplen}}1r]{% \aunderbrace brace script
        \aoverbrace
          [L1@{\hspace{-\bracecusplen}}1R000]% \aoverbrace brace script
          {\FnD}% stuff
          [1u13]% \aoverbrace script spec
          ^{2/5}% \aoverbrace upper script
      }
      [43d3]% \aunderbrace script spec
      _{3/5}$% \aunderbrace lower script
    
  \item
    \lstinline!\newbracespec{a}{@{\hspace{-.5\bracecusplen}}D}! \\
    \quad\lstinline!\newbracespec{z}{D@{\hspace{-.5\bracecusplen}}}! \\
    \quad\lstinline!\aunderbrace[l1r]{\FnD}! \\
    \quad\quad\lstinline![a1z]_{\text{\rlap{far left}} & \text{\llap{far right}}}!: \\
    \newbracespec{a}{@{\hspace{-.5\bracecusplen}}D}%
    \newbracespec{z}{D@{\hspace{-.5\bracecusplen}}}%
    $\aunderbrace[l1r]{\FnD}[a1z]_{\text{\rlap{far left}} & \text{\llap{far right}}}$
\end{itemize}

\begin{lrbox}{\codebox}\footnotesize\lstinline!\overbrace!\end{lrbox}% Capture code in \codebox
Here is a real-world example where ``breaking'' a brace across lines is required to indicate a continuous grouping of objects. This example\footnote{Taken from the question \href{http://tex.stackexchange.com/q/25510/5764}{\usebox{\codebox} split across multiple lines} on the TeX StackExchange network.} constructs two open-ended \lstinline!\aoverbrace!s that ``span'' multiple lines:

\begin{multline*}
  f(x) = a_0 + a_1 x + a_2 x^2 +
    \aoverbrace[L1U1@{~\ldots}]{a_3 x^3 + a_4 x^4 + \dots + 
      a_{i - 1} x^{i - 1} + \quad}^
      {\text{some text}} \\[\jot]
    \aoverbrace[@{\ldots~}1R]{\quad a_i x^i + a_{i + 1} x^{i + 1}} + 
      \dots + a_{n - 1} x^{n - 1}
\end{multline*}

\begin{lstlisting}
\usepackage{amsmath}% http://ctan.org/pkg/amsmath
%...
\begin{multline*}
  f(x) = a_0 + a_1 x + a_2 x^2 +
    \aoverbrace[L1U1@{~\ldots}]{a_3 x^3 + a_4 x^4 + \dots + 
      a_{i - 1} x^{i - 1} + \quad}^
      {\text{some text}} \\[\jot]
    \aoverbrace[@{\ldots~}1R]{\quad a_i x^i + a_{i + 1} x^{i + 1}} + 
      \dots + a_{n - 1} x^{n - 1}
\end{multline*}
\end{lstlisting}

As a final example, consider a brace that should include a dashed component. Using \lstinline|\newbracespec| one can define your own dashed component:
\begin{lstlisting}
\newbracespec{d}{%
  5@{\hspace{4pt}}1@{\hspace{4pt}}!{2em}@{\hspace{4pt}}1@{\hspace{4pt}}5%
}
\end{lstlisting}
\noindent and then use
\begin{lstlisting}
\[
  \aunderbrace[l*{3}{d}D*{3}{d}r]{\FnD}_{\text{What happened to the cat?}}
\]
\end{lstlisting}
\[
  \newbracespec{d}{1@{\hspace{4pt}}1@{\hspace{4pt}}!{1em}@{\hspace{4pt}}1@{\hspace{4pt}}1}
  \aunderbrace[l*{3}{d}D*{3}{d}r]{\FnD}_{\text{What happened to the cat?}}
\]

\section{Terms of reference}

This package originated from a question on the TeX StackEchange network called \href{http://tex.stackexchange.com/q/68526/5764}{Asymmetric overbrace}. Some code was taken from the \href{http://ctan.org/pkg/mathtools}{\sty{mathtools} package}.

This material is released under and subject to the \href{https://latex-project.org/lppl/}{LaTeX Project Public Licence}.

\section{Acknowledgements}

Thanks to Frank Mittlebach who stepped in and suggested an improvement in the original way \sty{abraces} functioned. Expansions included the use of \LaTeX3 command interface (via the \href{http://ctan.org/pkg/xparse}{\sty{xparse} package}).

\section{Change log}

\begin{itemize}
  \item v2.0 (2021-03-31) Major update
  
  \begin{itemize}
    \item
    The package now uses \href{http://ctan.org/pkg/xparse}{\sty{xparse}} for macro definitions.
    
    \item
    Included an automated way for handling elementary/default scripts. The placement of scripts can still be modified using the \lstinline|\bracescript| interface.
    
    \item
    The documentation was also updated to reflect the changes.
  \end{itemize}
  
  \item v1.0 (2012-08-31) Initial release
\end{itemize}

\end{document}
